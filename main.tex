\documentclass[1col]{ceurart}

\usepackage{listings}
\usepackage{xcolor}
\usepackage{amsmath} % Added for split environment

% Listing style for Turtle/SPARQL
\lstset{
    basicstyle=\ttfamily\small,
    breaklines=true,
    frame=single,
    backgroundcolor=\color{gray!10},
    captionpos=b,
    keywordstyle=\color{blue!60!black},
    stringstyle=\color{red!60!black},
    commentstyle=\color{green!40!black}
}

\begin{document}

\copyrightyear{2026}
\copyrightclause{Copyright for this paper by its authors. Use
  permitted under Creative Commons License Attribution 4.0
  International (CC BY 4.0).}
\conference{SWAT4HCLS 2026: Semantic Web Applications and Tools for
  Health Care and Life Sciences, February 16--19, 2026, Leiden, The
  Netherlands}

\title{Ontological modeling of dynamic biodiversity consensus}

\author[1]{Robert Hoehndorf}[%
    email=robert.hoehndorf@kaust.edu.sa,
    url=https://bio-ontology.org,
    orgname={King Abdullah University of Science and Technology (KAUST)},
    orgaddress={Computational Bioscience Research Center, Thuwal, Saudi Arabia}
]

\begin{abstract}
  The digitization of biodiversity in extreme environments, such as the Rub' al Khali (Empty Quarter), relies increasingly on citizen science platforms like iNaturalist. However, the data produced is not static; taxonomic identifications evolve through community consensus, creating a provenance challenge for the Semantic Web. We present a formal OWL-DL ontology rooted in the Semanticscience Integrated Ontology (SIO) to model this dynamic ecosystem. We distinguish between the TBox (defining the classes of consensus, evidence, and agents) and the ABox (the specific expedition observations). This distinction allows for reasoning over conflicting evidence and provides a rigorous structure for integrating "shifting" research-grade classifications into the Linked Open Data cloud.
\end{abstract}

\begin{keywords}
  OWL-DL \sep
  SIO \sep
  Biodiversity \sep
  Rub' al Khali \sep
  TBox/ABox
\end{keywords}

\maketitle

\section{Introduction}

The Rub' al Khali, the world's largest sand desert, represents a
significant data void in global biodiversity monitoring. To address
this, we established a digitization project on iNaturalist seeded by
research expeditions. While effective for data mobilization, the
platform's consensus mechanism, where an observation's identity
``flips'' based on user voting, presents a semantic challenge. Existing Darwin
Core mappings \cite{dwc} capture only the snapshot of the current
state, losing the history of disagreement essential for scientific
rigor. We propose a solution using OWL 2 DL (Web Ontology Language,
Description Logic) aligned with the Semanticscience Integrated
Ontology (SIO) \cite{sio}.

By strictly separating the ontological schema (TBox) from the instance
data (ABox), we enable automated reasoning to detect logical
inconsistencies in taxonomic assertions.


\section{Ontology design}

To model scientific disagreement while respecting biological
hierarchy, we treat taxonomy as a system of class subsumptions (for
hierarchy) and disjointness (for speciation). We separate epistemic
assertions (the act of claiming that an observed individual belongs to
a certain taxon) from the underlying reality (the fact that the
observed individual belongs to exactly one species).

\subsection{The identification process}

We model an identification not as a simple link, but as a reified
\texttt{sio:process}. This captures the temporal and agent-driven
nature of scientific naming, and allows us to add information to this
process. An \texttt{IdentificationAct}:
\begin{enumerate}
    \item Is performed by an Agent (\texttt{sio:has-agent}).
    \item Occurs at a specific time
      (\texttt{sio:has-time-boundary}\todo{Not this one, must be an interval}).
    \item Targets a specific Observation (\texttt{sio:has-target}).
    \item Outputs a Taxon determination (\texttt{sio:has-output}).
\end{enumerate}

\begin{equation}
\begin{split}
    \texttt{IdentificationAct} \sqsubseteq \ & \texttt{sio:process} \\
    & \sqcap \exists \texttt{sio:has-agent}.\texttt{sio:Agent} \\
    & \sqcap \exists \texttt{sio:has-target}.\texttt{Observation} \\
    & \sqcap \exists \texttt{sio:has-output}.\texttt{Taxon}
\end{split}
\end{equation}

\subsection{Handling Revisions and Consensus}
Taxonomic identification is inherently non-monotonic; agents can
revise their assertions based on new photographic evidence or
community discussion.  We utilize the SIO process model to represent
the lifecycle of an identification:
\begin{enumerate}
\item \textbf{Temporal Grounding:} Every identification act is
  assigned a strict timestamp using \texttt{sio:has-time-boundary}
  (SIO\_000681), anchoring the scientific opinion to a specific moment
  in the discussion.
\item \textbf{Revision History:} When an agent updates their
  determination (e.g., refining a Genus to a Species), the new act is
  explicitly linked to the prior act using
  \texttt{sio:is-successor-of} (SIO\_000630). This allows us to
  reconstruct the full trajectory of an agent's reasoning.
\item \textbf{Active vs. Historic Status:} We define an
  \texttt{ActiveIdentification} as any identification act that is not
  the target of a \texttt{sio:is-successor-of} relation. This
  distinction allows the reasoner to filter out superseded opinions
  and calculate consensus solely based on the current state of
  knowledge.
\end{enumerate}

Formally, we define the set of active identifications ($I_{active}$)
for a given observation ($Obs$) as:
\begin{equation}
  I_{active}(Obs) = \{ i \in I \mid \texttt{hasTarget}(i, Obs) \land
  \neg \exists j ( \texttt{isSuccessorOf}(j, i) ) \} 
\end{equation}

By restricting the conflict detection rules (defined previously) to
operate strictly over $I_{active}$, we ensure that historical
disagreements, which have since been resolved by the agents
themselves, do not trigger false positive conflict flags.

\subsection{Epistemic conflict detection}

We distinguish between ontological inconsistency (which results in a
formal contradiction and will prevent any automated reasoning) and
epistemic conflict (disagreement between agents). We assume the
\texttt{Observation} refers to a single biological reality (one
individual, belonging to one species, is observed). However, rather
than forcing the Observation to instantiate conflicting classes, we
detect conflict at the assertion level using SWRL rules.

We employ punning to treat taxa as individuals in the SWRL rules. We
define an auxiliary property, \texttt{isIncompatibleWith}, which links
any two taxa that are disjoint in the TBox (e.g., distinct species).
\begin{equation}
    \texttt{Taxon}_A \sqcap \texttt{Taxon}_B \sqsubseteq \bot \implies
    \texttt{isIncompatibleWith}(\texttt{Taxon}_A, \texttt{Taxon}_B)
\end{equation}

We then define the conflict rule: if an observation is the target of
two \textit{active} identification acts that output incompatible taxa,
the observation is classified as a \texttt{ConflictingObservation}.

\begin{equation}
  \begin{split}
    \texttt{IdentificationAct}(?a1) \wedge \texttt{hasTarget}(?a1, ?obs) \wedge \texttt{hasOutput}(?a1, ?t1) \\
    \wedge \ \texttt{IdentificationAct}(?a2) \wedge \texttt{hasTarget}(?a2, ?obs) \wedge \texttt{hasOutput}(?a2, ?t2) \\
    \wedge \ \texttt{isIncompatibleWith}(?t1, ?t2) \\
    \rightarrow \texttt{ConflictingObservation}(?obs)
  \end{split}
\end{equation}

This approach allows the ABox to contain contradictory evidence
(reflecting reality) while enabling the system to programmatically
flag the observation for review.




\section{Results: Instantiation (ABox)}

The ABox contains the assertions derived from the Rub' al Khali project (\url{https://www.inaturalist.org/projects/rub-al-khali}).

\subsection{Data Provenance}
We extracted data from the iNaturalist project, which serves as a central aggregation point for biodiversity in the region. As of February 2026, the dataset comprises \textbf{[INSERT OBS COUNT]} observations and \textbf{[INSERT ID COUNT]} identifications. These records define a baseline for the Empty Quarter's flora and fauna. Notably, approximately \textbf{[25\%]} of these observations have achieved "Research Grade" status via community consensus.

We employ an ETL pipeline to convert the iNaturalist API JSON response into RDF triples, linking the data to the Global Biodiversity Information Facility (GBIF, \url{https://www.gbif.org}) backbone where applicable.

\subsection{Example Instance}
Below is a simplified Turtle representation of a specific observation (\texttt{obs:1942}) showing the TBox instantiation.

\begin{lstlisting}[language=SPARQL, caption=ABox instance of a disputed identification]
@prefix sio: <http://semanticscience.org/resource/> .
@prefix ex: <http://example.org/rub-al-khali/> .
@prefix dwc: <http://rs.tdwg.org/dwc/terms/> .

# The Observation Instance
ex:obs_1942 a sio:observation ;
    rdfs:label "Cornulaca arabica observation" ;
    dwc:decimalLatitude "20.123" ;
    dwc:decimalLongitude "50.456" .

# Identification Act 1 (The Author)
ex:id_act_1 a ex:IdentificationAssertion ;
    sio:has-agent ex:agent_rhoehndorf ;
    sio:has-target ex:obs_1942 ;
    sio:has-output ex:taxon_cornulaca_arabica .

# Identification Act 2 (Community Disagreement)
ex:id_act_2 a ex:IdentificationAssertion ;
    sio:has-agent ex:agent_community_user ;
    sio:has-target ex:obs_1942 ;
    sio:has-output ex:taxon_cornulaca_monacantha .

# Resulting Inferred Status (via Reasoner)
# ex:obs_1942 rdf:type ex:ConflictingObservation .
\end{lstlisting}

\section{Discussion and Conclusion}

By distinguishing between the TBox (the logic of consensus) and the ABox (the expedition data), we gain several advantages over flat metadata schemas such as Darwin Core archives:
\begin{enumerate}
    \item \textbf{Monotonicity:} New identifications are simply added to the ABox. The reasoner infers the change in status (e.g., from Research Grade to Conflict) without deleting prior assertions.
    \item \textbf{Querying Disagreement:} We can write SPARQL queries to extract only those instances in the Empty Quarter where specific experts disagree, prioritizing them for DNA barcoding.
\end{enumerate}

This approach ensures that the digital representation of the Rub' al Khali's flora is not just a static archive, but a living knowledge graph that accurately reflects the scientific process of identification.

\begin{thebibliography}{4}

\bibitem{sio}
Dumontier, M., et al.: The Semanticscience Integrated Ontology (SIO) for biomedical research and knowledge discovery. Journal of Biomedical Semantics 5, 14 (2014).

\bibitem{inat}
iNaturalist: A Community for Naturalists. \url{https://www.inaturalist.org}.

\bibitem{dwc}
Wieczorek, J., et al.: Darwin Core: An Evolving Community-Developed Biodiversity Data Standard. PLoS ONE 7(1), e29715 (2012). \url{https://dwc.tdwg.org}

\bibitem{owl}
Hitzler, P., et al.: OWL 2 Web Ontology Language Primer. W3C Recommendation (2009).

\end{thebibliography}

\end{document}