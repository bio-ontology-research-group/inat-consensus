\documentclass[1col]{ceurart}

\usepackage{listings}
\usepackage{xcolor}
\usepackage{amsmath}

% Listing style for Turtle/SPARQL
\lstset{
    basicstyle=\ttfamily\small,
    breaklines=true,
    frame=single,
    backgroundcolor=\color{gray!10},
    captionpos=b,
    keywordstyle=\color{blue!60!black},
    stringstyle=\color{red!60!black},
    commentstyle=\color{green!40!black}
}

\begin{document}

\copyrightyear{2026}
\copyrightclause{Copyright for this paper by its authors. Use
  permitted under Creative Commons License Attribution 4.0
  International (CC BY 4.0).}
\conference{SWAT4HCLS 2026: Semantic Web Applications and Tools for
  Health Care and Life Sciences, February 16--19, 2026, Leiden, The
  Netherlands}

\title{A provenance-aware semantic model for evolving biodiversity consensus}



\author[1]{Chuck Lee}[%
  email=chuck.lee@example.org,
]
\address[1]{Bio2Vec Research Group, KAUST, Thuwal, Saudi Arabia}


\begin{abstract}
  Community-driven biodiversity platforms generate dynamic datasets where taxonomic identifications evolve continuously through user interaction. However, current standards treat these identifications as static attributes, discarding the rich provenance of refinement, disagreement, and error correction. Here, we present a semantic methodology to model the trajectory of taxonomic change using OWL 2 DL and the PROV-O ontology. We reify the identification act as a provenance-generating activity, allowing us to reconstruct the causal chain of classification states. By aligning this history with a "Deep Disjointness" TBox, we enable the automated classification of provenance transitions into semantic refinements, coarsenings, or epistemic conflicts. We validate this method using a case study of 227 observations from an extreme environment dataset, demonstrating how 11 instances of cryptic disagreement were algorithmically recovered from the provenance graph.
\end{abstract}

\begin{keywords}
  Provenance \sep
  PROV-O \sep
  Ontology \sep
  Taxonomic change \sep
  Conflict detection
\end{keywords}

\maketitle

\section{Introduction}

In the domain of biodiversity informatics, an "identification" is rarely a final truth; rather, it is a mutable hypothesis subject to revision. Citizen science platforms like iNaturalist facilitate this evolution by allowing community members to propose, vote on, and supersede taxonomic claims [1]. While this process generates high-quality consensus data, the *history* of how that consensus was reached is often lost in translation to standard formats like Darwin Core [2].

From a Semantic Web perspective, this represents a provenance failure. When an observation changes from \textit{Genus A} to \textit{Species A.1}, it is a semantic refinement. When it changes from \textit{Genus A} to \textit{Genus B}, it is an epistemic conflict. Capturing these distinctions requires a rigorous provenance model.

Here, we propose a methodology that aligns the Semanticscience Integrated Ontology (SIO) [3] with the W3C PROV-O standard [4] to model the "Causal Chain" of identification. We demonstrate that by explicitly modeling the derivation history of taxonomic assertions, we can algorithmically reconstruct the community's reasoning process and detect latent disagreements that persist beneath the consensus.

\section{Methodology}

Our approach shifts the modeling paradigm from "What is this organism?" to "How did this organism come to be classified as such?" We achieve this through three methodological components: the PROV-O alignment, the Causal Chain reconstruction, and the Semantic Transition logic.

\subsection{Alignment with PROV-O}
We map the native iNaturalist data model to PROV-O concepts to establish a standardized provenance graph:
\begin{enumerate}
    \item **Identification Act as Activity:** We model the act of identifying an organism as a `prov:Activity` (specifically, `sio:process`). This activity is associated with a start time and an agent (`prov:wasAssociatedWith`).
    \item **Taxon as Entity:** The output of the identification is a taxonomic concept, modeled as a `prov:Entity` (or `sio:Taxon`).
    \item **Derivation:** When a user proposes an identification that supersedes a previous one, we model this as a `prov:wasInformedBy` relationship between the activities, or a `sio:is-successor-of` link between the identification acts.
\end{enumerate}

This structure ensures that every classification state is linked to its predecessor, creating a directed acyclic graph (DAG) of opinion evolution for each observation.

\subsection{The Deep Disjointness TBox}
To interpret the meaning of a change between two provenance states, the system requires a rigorous TBox. We construct a hybrid ontology combining local taxonomic trees with the NCBI Taxonomy [5]. Crucially, we apply a "Deep Disjointness" algorithm that enforces `owl:disjointWith` assertions between all non-nested taxa. This allows the reasoner to distinguish between a change that narrows a hypothesis (e.g., *Bird* $\rightarrow$ *Eagle*) and one that contradicts it (e.g., *Eagle* $\rightarrow$ *Hawk*).

\subsection{Classifying semantic transitions}
By querying the provenance graph against the TBox, we classify the transition $T$ between an old identification $Id_{old}$ (asserting Taxon $A$) and a new identification $Id_{new}$ (asserting Taxon $B$) into three categories:

\begin{enumerate}
    \item **Refinement:** If $B \sqsubseteq A$, the community has increased precision (e.g., *Animal* $\rightarrow$ *Fox*).
    \item **Coarsening:** If $A \sqsubseteq B$, the community has retreated to ambiguity (e.g., *Fox* $\rightarrow$ *Canid*).
    \item **Epistemic Conflict:** If $A \sqcap B \sqsubseteq \bot$ (i.e., they are disjoint), the change represents a fundamental disagreement or correction (e.g., *Fox* $\rightarrow$ *Cat*).
\end{enumerate}

\section{Case Study: Rub' al Khali}

To validate this methodology, we applied the pipeline to a dataset from the Rub' al Khali (Empty Quarter) desert, a region characterized by sparse data and high taxonomic uncertainty.

\subsection{Provenance graph reconstruction}
The dataset contained 227 observations. Our workflow generated a provenance graph containing 546 distinct `prov:Activity` instances (Identification Acts). The system successfully reconstructed the causal chains for all observations, linking initial field IDs to subsequent expert revisions.

\subsection{Conflict recovery}
Using the semantic transition logic, we isolated 11 instances of **Epistemic Conflict** where the provenance chain terminated in a state of active disagreement (i.e., multiple agents asserting disjoint taxa without resolution). These conflicts were not explicitly flagged in the source metadata but were emergent properties of the provenance graph when reasoned over with the Deep Disjointness TBox.

For instance, Observation \texttt{obs\_321457657} revealed a "Flip" transition where an initial identification of *Stenodactylus* (Gecko) was contested by a later claim of *Bunopus*, a disjoint genus. The PROV-O model captured this not just as a current state, but as a historical debate between two specific agents.

\subsection{Environmental context}
To further contextualize these provenance chains, we linked the activities to their environmental setting using the Environment Ontology (ENVO) [6]. We found that 100\% of the activities occurred within `ENVO:00000115` (Sand Desert), providing a semantic backdrop that can be used to query for environmentally-correlated disagreement rates in future studies.

\section{Conclusion}

We presented a provenance-centric method for modeling biodiversity data that elevates taxonomic history from metadata to a first-class citizen of the ontology. By aligning identification acts with PROV-O and validating transitions against a rigorous TBox, we demonstrated the ability to algorithmically recover the semantic history of community consensus. This approach offers a standardized pathway for integrating dynamic citizen science data into the Semantic Web without losing the critical context of scientific debate.

\begin{thebibliography}{6}

\bibitem{inat}
iNaturalist. Available from: \url{https://www.inaturalist.org}. Accessed 2026-02-16.

\bibitem{dwc}
Wieczorek J, et al. Darwin Core: An Evolving Community-Developed Biodiversity Data Standard. PLoS ONE. 2012;7(1):e29715.

\bibitem{sio}
Dumontier M, et al. The Semanticscience Integrated Ontology (SIO) for biomedical research and knowledge discovery. J Biomed Semantics. 2014;5:14.

\bibitem{provo}
Lebo T, et al. PROV-O: The PROV Ontology. W3C Recommendation. 2013.

\bibitem{ncbi}
Federhen S. The NCBI Taxonomy database. Nucleic Acids Res. 2012;40(Database issue):D136-D143.

\bibitem{envo}
Buttigieg PL, et al. The environment ontology: contextualising biological and biomedical entities. J Biomed Semantics. 2013;4(1):43.

\end{thebibliography}

\end{document}